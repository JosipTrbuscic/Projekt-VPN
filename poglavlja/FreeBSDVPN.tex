Ovo poglavlje če uključivati postavljanje VPN-a na FreeBSD inačici operacijskog
sustava BSD. 

\subsection{Postavljanje FreeBSD poslužitelja}
    Za pristupanje udaljenom poslužitelju koristit ćemo program i istoimeni
    program ssh. Kako sada prvi puta pristupamo poslužitelju jedini korisnik
    koji postoji je root. Root je korisnik na unixoidima koji može izvršiti
    svaku naredbu i pristupiti svakoj datoteci. Njemu pristupamo naredbom
    
    \code{\$ ssh root@139.59.159.111} \\

    Kada smo se prijavili na poslužitelja prvu stavr koju moramo napraviti je
    ažurirati sustav. To ćemo napraviti koristeći FreeBSD-ov upravitelj
    paketima \code{pkg} i njegove naredbe \code{update} i \code{upgrade}.

    \code{\# pkg update} \\
    \code{\# pkg upgrade} \\

    Sada možemo instalirati openvpn.

    \code{\# pkg install openvpn} \\

    Konfiguracijske datoteke ćemo smjestiti u direktorij
    \code{/usr/local/etc/openvpn} koji prvo moramo napraviti.

    \code{\# mkdir /usr/local/etc/openvpn} \\

    Openvpn nudi predloške konfiguracijskih datoteka stoga ćemo ih kopirati u
    naš direktorij

    \noindent
    \code{\# cp
    /usr/local/share/examples/openvpn/sample-config-files/server.conf
    \textbackslash} \\
    \code{\-\ \-\ \-\ \-\ \-\ /usr/local/etc/openvpn/openvpn.conf}

    \subsubsection{Konfiguracija}
        Kako bi mogli zaštititi našu vezu potrebno je kriptirati sav promet
        između poslužitelja i klijenta. Koristit ćemo asimetričnu enkripciju
        koja za što su nam potrebni parovi privatnih i javnih ključeva. OpenVPN
        dolazi sa alatom Easy-RSA koji će nam poslužiti za izgradnju
        infrastrukture javnog ključa (\textit{engl. PKI - public key
        infrastructure}). PKI služi kako bi se javni ključevi povezali s
        pripadajućim osobama ili oragnizacijama. Proces povezivanja izvršava
        tijelo za certificiranje (\textit{engl. CA - certification authority}.
        CA također potvrđuje privada li javni ključ osobi navedenoj u
        certifikatu.

        Kako je Easy-RSA omotač oko složene programske knjižnice OpenSSL ona
        nam je jedini preduvjet te ćemo ju instalirati naredbom

        \noindent
        \code{\# pkg install openssl} \\

        Nakon toga ćemo kopirati \code{easy-rsa} skriptu u naš direktorij sa svom
        konfiguracijom.

        \noindent
        \code{\# cp -r /usr/local/share/easy-rsa
        /usr/local/etc/openvpn/easy-rsa} \\

        Sada ćemo se premjestiti u Easy-RSA direktoriji i urediti njegovu
        konfiguracijusku datoteku \code{vars}.

        \noindent
        \code{\# cd /usr/local/etc/openvpn/easy-rsa} \\
        \code{\# vim vars} \\

        U nastavku su navedena polja koja je potrebno izmjeniti:
        
        \noindent
        \code{set\_var EASYRSA\_REQ\_COUNTRY   \-\ "<ZEMLJA>"} \\
        \code{set\_var EASYRSA\_REQ\_PROVINCE  "<ZUPANIJA>"} \\
        \code{set\_var EASYRSA\_REQ\_CITY      \-\ \-\ \-\ \-\ "<GRAD>"} \\
        \code{set\_var EASYRSA\_REQ\_ORG       \-\ \-\ \-\ \-\ \-\ "<ORGANIZACIJA>"} \\
        \code{set\_var EASYRSA\_REQ\_EMAIL     \-\ \-\ \-\ "<EMAIL>"} \\
        \code{set\_var EASYRSA\_REQ\_OU        \-\ \-\ \-\ \-\ \-\ \-\  "<ORGANIZACIJSKA JEDINICA>"} \\
        \code{set\_var EASYRSA\_KEY\_SIZE      \-\ \-\ \-\ \-\ <broj> \# duljina rsa ključa u
        bitovima} \\
        \code{set\_var EASYRSA\_CA\_EXPIRE     \-\ \-\ \-\ <broj> \# trajanje CA ključa u
        danima} \\
        \code{set\_var EASYRSA\_CERT\_EXPIRE   \-\ <broj> \# trajanje certifikata u
        danima} \\

        Kako je \code{easy-rsa} skripta pisana za ljusku sh, dok
        FreeBSD koristi csh potrebno je naredbom \code{sh} 
        pokrenuti sh ljsuksu. Sada možemo inicijalizirati PKI

        \code{\# ./easy-rsa.real init-pki}
        
        \noindent
        nakon čega ćemo stvoriti CA 

        \noindent
        \code{\# ./easy-rsa.real build-ca}
        
        \noindent
        Ovom naredbom smo stvorili par ključeva koji ćemo koristiti za
        potpisivanje izdanih certifikata. 

        \noindent
        Sada ćemo generirati serverov certifikat naredbom

        \noindent
        \code{\# ./easy-rsa.real build-server-full <ime-server> nopass }

        \noindent
        gdje je \code{<ime-server>} ime certifikata, a s \code{nopass} opcijom ćemo
        generirati nešifrirani ključ kako bi mogli automatski pokrenuti OpenVPN
        uslugu prilikom pokretanja sustava bez upisivanja lozinke ključa.

        Na sličan način ćemo generirati klijentove certifikate

        \noindent
        \code{\# ./easy-rsa.real build-client-full <ime-klijent>}
        
        \noindent
        Za šifriranje same poruke koristit ćemo simetričan ključ generiran
        Diffie-Hellman razmjenom. Za to su nam potrebni Diffie-Hellman
        parametri koje stvaramo naredbom

        \noindent
        \code{\# ./easyrsa.real gen-dh}

        \noindent
        Kako bi dodatno osigurali integritet podataka kkk

        \noindent
        Do sada smo sve naredbe izvršavali na poslužitelju te smo generirali
        velik broj datoteka od kojih ćemo neke morati premjestiti na klijentsko
        računalo. Kako bi znali koje datoteke premejstiti potrebno je razumjeti
        čemu svaka od njih služi. Kako su sve datoteke stvorene u \code{pki}
        direktoriju pretpostavit ću da smo se u njega pozicionirali.

        \begin{itemize}
        \item \code{ca.crt} - certifikat koji se koristi za validaciju ostalih
        certifikata, potrebno ga je kopirati na poslužitelja i sve klijente
        \item \code{ca.key} - ključ koji CA koristi za izdavanje certifikata
        \item \code{reqs/} - direktorij koji sadrži zahtjeve za izdajom
        certifikata 
        \item \code{issued/<ime-server>.crt} - certifikat servera koji služi za
        provjeru potpisa na poruci, potrebno ga je prebaciti na poslužitelja
        \item \code{private/<ime-server>.key} - privatni ključ polužitelja
        koji se koristi za potpisivanje poruke, potrebno ga je prebaciti na
        poslužitelja
        \item \code{issued/<ime-klijent>.crt} - certifikat klijenta koji služi za
        provjeru potpisa na poruci, potrebno ga je prebaciti na klijentsko
        računalo
        \item \code{private/<ime-klijent>.key} - privatni ključ klijenta
        koji se koristi za potpisivanje poruke, potrebno ga je prebaciti na
        klijentsko računalo
        \item \code{dh.pem} - Diffie Hellman parametri, potrebno ih je prebaciti na
        poslužitelja
        \end{itemize}

        Ključeve poslužitelja ćemo premjestiti u poseban direktorij

        \noindent
        \code{\# mkdir /usr/local/etc/openvpn/keys} \\
        \code{\# cp pki/dh.pem \textbackslash} \\
        \code{\-\ \-\ \-\ \-\ \-\ pki/ca.crt \textbackslash} \\
        \code{\-\ \-\ \-\ \-\ \-\ pki/issued/<ime-server>.crt \textbackslash} \\
        \code{\-\ \-\ \-\ \-\ \-\ pki/private/<ime-server>.key \textbackslash} \\
        \code{\-\ \-\ \-\ \-\ \-\ /usr/local/etc/openvpn/keys} \\

        \noindent
        Prije nego što počnemo konfigurirati klijentsko računalo, potrebno je u
        konfiguraciji poslužitelja navesti putanje do certifikata, ključeva i 
        parametara. To ćemo napraviti u datoteci \code{openvpn.conf} koju smo
        na samom početku kopirali u \code{/usr/local/etc/openvpn}.
        
        \noindent
        \code{vim /usr/local/etc/openvpn/openvpn.conf} \\
        
        Potrebno je urediti sljedeće linije

        \noindent
        \code{ca /usr/local/etc/openvpn/keys/ca.crt} \\
        \code{cert /usr/local/etc/openvpn/keys/<ime-server>.crt} \\
        \code{key /usr/local/etc/openvpn/keys/<ime-server>.key} \\
        \code{dh /usr/local/etc/openvpn/keys/dh.pem} \\ 

        Sada možemo postaviti klijentsko računalo. Kako smo već pripremili
        većinu klijentovih datoteka na poslužitelju, potrebno ih je kopirati.
        Radi se o osjetljivim datotekama stoga nam je potreban siguran način
        slanja datoteka preko mreže za što ćemo koristiti program
        \textit{secure copy}. Nakon što instaliramo openvpn isto kao i na
        poslužiteljskom računalu možemo kopirati predložak konfiguracije

        \code{\# cp
        /usr/local/share/examles/openvpn/sample-config-files/client.config
        \textbackslash} \\
        \code{\-\ \-\ \-\ \-\ \-\ /usr/local/etc/openvpn/openvpn.conf}

        \noindent
        Također stvorit ćemo direktorij u koji ćemo spremiti ključeve i
        certifikate

        \code{\# mkdir /usr/local/etc/openvpn/keys}\\

        \noindent
        Sada možemo kopirati potrebne datoteke sa poslužitelja

        \noindent
        \code{\$ scp
        root@<ip-server>:/usr/local/etc/openvpn/easy-rsa/pki/ca.crt keys } \\
        \code{\$ scp
        root@<ip-server>:/usr/local/etc/openvpn/easy-rsa/pki/issued/<ime-klijent>.crt
        keys} \\
        \code{\$ scp
        root@<ip-server>:/usr/local/etc/openvpn/easy-rsa/pki/private/<ime-klijent>.key}
        \\

        \noindent
        U konfiguraciji (\code{openvpn.conf}) osim putanja do certifikata i
        ključeva potrebno je unjeti ip adresu poslužitelja

        \noindent
        \code{remote <ip-server> 1194} \\
        \code{ca /usr/local/etc/openvpn/keys/ca.crt} \\
        \code{cert /usr/local/etc/openvpn/keys/<ime-server>.crt} \\
        \code{key /usr/local/etc/openvpn/keys/<ime-server>.key} \\
