\bigbreak
\subsection*{Što je SoftEther VPN?}

\begin{wrapfigure}{r}{0.5\textwidth} 
     \centering
     \includegraphics[width=0.5\textwidth]{SoftEther/SoftEtherLogo}
	\caption{Službeni logo SoftEther VPN-a}
\end{wrapfigure}
\hspace{0.5cm}
SoftEther VPN je besplatan višeplatfo- rmski program otvorenog koda koji podržava korištenje različitih VPN protokola. Program je nastao 2013. godine kao akademski projekt na sveučilištu u Tsukubi i podržan je na različitim operacijskim sustavima kao što su Linux, FreeBSD, Mac, Solaris i Windows za koji je u ovom poglavlju prikazan postupak postavljanja i uporabe.
\smallbreak
Zbog toga što je SoftEther otvorenog koda može ga bilo tko koristiti za osobnu ili komercijalnu uporabu bez plaćanja. 
\smallbreak
SoftEther VPN koristi HTTPS preko SSL (Secure Sockets Layer) %https://searchsecurity.techtarget.com/definition/SSL-VPN
protokola kako bi omogućio siguran prijenos kriptiranih podataka preko Interneta. Uz njega su podržani unutar programa i ostali poznatiji protokoli kao što su OpenVpn, IPsec, L2TP, ...\\
Unutar programa sve postavke detaljno su objašnjene i mogu se podesiti korištenjem grafičkog sučelja što ovaj program čini jednostavnim za uporabu.

\FloatBarrier

\bigbreak
\subsection*{Instalacija SoftEther servera}
\hspace{0.5cm}
Za početak potrebno je preuzeti instalaciju VPN servera sa službene stranice SoftEthera:\\ https://www.softether.org
\begin{figure}[h!]
	\centering
     \includegraphics[width=0.6\textwidth]{SoftEther/korak1}
\end{figure}
\FloatBarrier
Odabirom Download iz izborne trake prikazuje se stranica s ponuđenim vezama za preuzimanje.
\begin{figure}[h!]
     \centering
     \includegraphics[width=0.6\textwidth]{SoftEther/korak2}
\end{figure}
\FloatBarrier
Sljedeći isječak prikazuje stranicu koja se otvori odabirom prve veze. Na stranici se nalaze izborni okviri u kojima je potrebno odabrati željeni program. Za preuzimanje VPN servera potrebno je odabrati postavke prikazane na sljedećem isječku te odabrati prvu vezu za početak preuzimanja.
\begin{figure}[h!]
     \centering
     \includegraphics[width=0.6\textwidth]{SoftEther/korak6}
\end{figure}
\FloatBarrier
Nakon preuzimanja i pokretanja instalacije otvara se sljedeći prozor u kojemu se predlaže instaliranje prvog ponuđenog jer nudi potpunu instalaciju.
\begin{figure}[h!]
     \centering
     \includegraphics[width=0.6\textwidth]{SoftEther/korak7}
\end{figure}
\FloatBarrier
Nakon uspješne instalacije prikazuje se sljedeći okvir u kojem još nema niti jednog servera. Dodavanje servera započinje se odabirom ``New Setting''.
\begin{figure}[h!]
     \centering
     \includegraphics[width=0.6\textwidth]{SoftEther/korak8}
\end{figure}
\FloatBarrier
Stvaranje servera započinje se upisom željenog imena servera u polje ``setting name'' i upisom vlastite IP adrese preko koje je trenutno računalo spojeno na Internet. Upute za pronalazak IP adrese mogu se naći na kraju ovog poglavlja. Preporuka je dodati lozinku za pristup serveru zbog dodatnog osiguranja u polje ``password''.
\begin{figure}[h!]
     \centering
     \includegraphics[width=0.6\textwidth]{SoftEther/korak9}
\end{figure}
\FloatBarrier
U tablici servera vidimo da je dodan novi kojega je sada potrebno konfigurirati odabirom ``Connect'' opcije.
\begin{figure}[h!]
     \centering
     \includegraphics[width=0.6\textwidth]{SoftEther/korak10}
\end{figure}
\FloatBarrier
Kako bi se druga računala uspjela povezati s napravljenim serverom, potrebno je dodati virtualno čvorište odabirom opcije ``Create a Virtual Hub''.
\begin{figure}[h!]
     \centering
     \includegraphics[width=0.6\textwidth]{SoftEther/korak11}
\end{figure}
\FloatBarrier
Virtualnom čvorištu postavljamo proizvoljno ime te dodajemo lozinku zbog dodatne sigurnosti.
\begin{figure}[h!]
     \centering
     \includegraphics[width=0.6\textwidth]{SoftEther/korak12}
\end{figure}
\FloatBarrier
Sada se može vidjeti novo dodano čvorište u tablici.
\begin{figure}[h!]
     \centering
     \includegraphics[width=0.6\textwidth]{SoftEther/korak13}
\end{figure}
\FloatBarrier
Sljedeći je korak odrediti tko se sve može povezati na naš server a to se radi odabirom gumba ``Menage Virtual Hub''.
\begin{figure}[h!]
     \centering
     \includegraphics[width=0.6\textwidth]{SoftEther/korak14}
\end{figure}
\FloatBarrier
Na ovom prozoru odabiremo ``Menage Users''.
\begin{figure}[h!]
     \centering
     \includegraphics[width=0.4\textwidth]{SoftEther/korak15}
\end{figure}
\FloatBarrier
Sada dodajemo korisnika kojem ćemo dati proizvoljno ime (u ovim uputama nazvan je korisnik klijent1 i u svim koracima gdje se to ime pojavi vama će se pojaviti vaše odabrano ime). Kako bi smanjili mogućnost zlouporabe VPN-a, odabiremo mogućnost prijave klijenta uporabom našeg certifikata i lozinke. Zbog toga odabiremo ``Create Certificate''.
\begin{figure}[h!]
     \centering
     \includegraphics[width=0.6\textwidth]{SoftEther/korak16}
\end{figure}
\FloatBarrier
U sljedećim je poljima moguće detaljno odrediti opis stvorenog klijenta kao i vrijeme njegovog postojanja.
\begin{figure}[h!]
     \centering
     \includegraphics[width=0.6\textwidth]{SoftEther/korak17}
\end{figure}
\FloatBarrier
Kada nam se otvori ovaj prozor postavit ćemo lozinku kojom će se naš klijent prijavljivati na server i koja će samo njemu biti poznata.
\begin{figure}[h!]
     \centering
     \includegraphics[width=0.6\textwidth]{SoftEther/korak18}
\end{figure}
\FloatBarrier
Nakon potvrde nastaju dvije datoteke: jedna je .cer a druga je .key i obje su neophodne za prijavu na naš server stoga ih mi moramo spremiti i prebaciti na računala koja će se htjeti povezati na server. Povezivanje na server objašnjeno je u jednom od sljedećih dijelova poglavlja.
\begin{figure}[h!]
     \centering
     \includegraphics[width=0.3\textwidth]{SoftEther/korak20}
\end{figure}
\FloatBarrier
Nakon potvrde vidljiv je korisnik koji se može spojiti na naš server. Moguće je naravno dodavanje više različitih korisnika i brisanje istih.
\begin{figure}[h!]
     \centering
     \includegraphics[width=0.6\textwidth]{SoftEther/korak19}
\end{figure}
\FloatBarrier

\newpage
\subsection*{Instalacija SoftEther klijenta}
\hspace{0.5cm}
Za razliku od instalacije i konfiguracije servera, instalacija je SoftEther klijenta jednostavnija. Prvi je korak preuzimanje instalacije sa službene stranice SoftEthera:\\ https://www.softether.org
\begin{figure}[h!]
	\centering
     \includegraphics[width=0.6\textwidth]{SoftEther/korak1}
\end{figure}
\FloatBarrier
Odabirom Download iz izborne trake prikazuje se stranica s ponuđenim vezama za preuzimanje.
\begin{figure}[h!]
     \centering
     \includegraphics[width=0.6\textwidth]{SoftEther/korak2}
\end{figure}
\FloatBarrier
Sljedeći isječak prikazuje stranicu koja se otvori odabirom prve veze. Na stranici se nalaze izborni okviri u kojima je potrebno odabrati željeni program. Za preuzimanje VPN klijenta potrebno je odabrati postavke prikazane na sljedećem isječku te odabrati prvu vezu za početak preuzimanja.
\begin{figure}[h!]
     \centering
     \includegraphics[width=0.6\textwidth]{SoftEther/korak3}
\end{figure}
\FloatBarrier
Nakon završetka preuzimanja i pokretanja instalacije prikazuje se sljedeći prozor. Preporuka je odabrati prvo ponuđeno jer nudi potpunu instalaciju programa.
\begin{figure}[h!]
     \centering
     \includegraphics[width=0.6\textwidth]{SoftEther/korak4}
\end{figure}
\FloatBarrier
Ukoliko je instalacija uspješno završena prikazuje se sljedeći prozor.
\begin{figure}[h!]
     \centering
     \includegraphics[width=0.6\textwidth]{SoftEther/korak5}
\end{figure}
\FloatBarrier

\newpage
\subsection*{Povezivanje klijenta sa SoftEther serverom}
\hspace{0.5cm}
Kako bi se povezali uspješno s napravljenim serverom, potrebno je pokrenuti aplikaciju SoftEether VPN Client i odabrati opciju dodavanja novog VPN-a. Ako nije postavljen virtualni mrežni adapter, kao što je prikazano, potrebno je stvoriti novi. Prikazano je stvaranje VPN adaptera.
\begin{figure}[h!]
     \centering
     \includegraphics[width=0.6\textwidth]{SoftEther/korak21}
\end{figure}
\FloatBarrier
Nakon stvaranja adaptera moguće je dodati server na koji se želimo povezati. Na slici je prikazano stvaranje veze koja se zove VPN. Slično kao i kod stvaranja servera, potrebno je upisati IP adresu preko koje se može serveru pristupiti u polje ``Host name''. Nakon upisa IP adrese aplikacija dohvaća portove na koje moguće spojiti. Izbor je nekog od ponuđenih portova proizvoljan kao i postojećih virtualnih mrežnih adaptera. Pošto smo prilikom stvaranja korisnika servera odabrali da se on može prijaviti samo uporabom certifikata i pripadnog ključa, potrebno je stvorene datoteke ``klijent1.cer'' i ``klijent1.key'' prebaciti na računalo s kojeg se pokušava povezati na server. Učitavanje certifikata i ključa u aplikaciju obavlja se odabirom opcije ``specify client certificate''.
\begin{figure}[h!]
     \centering
     \includegraphics[width=0.6\textwidth]{SoftEther/korak22}
\end{figure}
\FloatBarrier
Nakon učitavanja datoteka prikazuje se prozor na sljedećem isječku u koji se upisuje lozinka koju smo postavili prilikom stvaranja klijenta.
\begin{figure}[h!]
     \centering
     \includegraphics[width=0.6\textwidth]{SoftEther/korak23}
\end{figure}
\FloatBarrier
Ako smo učitali ispravni certifikat i unijeli ispravnu lozinku, tada će se prikazati prozor koji prikazuje spajanje na server.
\begin{figure}[h!]
     \centering
     \includegraphics[width=0.5\textwidth]{SoftEther/korak24}
\end{figure}
\FloatBarrier

\FloatBarrier

\subsection*{Provjera vlastite IP adrese}
\hspace{0.5cm}
Kako bi server bio uspješno uspostavljen, potrebna mu je IP adresa dodijeljena računalu na kojem se nalazi. Najbrži način na koji se ona može odrediti jest otvaranje naredbenog retka i upis naredbe IPCONFIG. Rezultat te naredbe bit će prikaz mrežnih postavki za trenutno aktivne mrežne adaptere.
Crvenom je strjelicom označena IP adresa na trenutno aktivnom adapteru.
\begin{figure}[h!]
     \centering
     \includegraphics[width=0.6\textwidth]{SoftEther/IP1}
\end{figure}
\FloatBarrier

% za usporedbu === https://www.softether.org/@api/deki/files/12/=1.3.jpg
% pomoć pri instalaciji === https://www.youtube.com/watch?v=VbvRhPqNCsk