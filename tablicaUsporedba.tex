Tijekom projekta shvatili smo da iako postoji velik broj VPN tehnologija
većina ih temelji na istom skupu protokola, a razlikuju se prema sučelju koje
nude korisniku. Za prosječnog korisnika koji se ne bavi administracijom većih
sustava te VPN koristi samo za osobnu uporabu svakako preporučamo tehnologiju
koja nudi grafičko sučelje i skriva većinu implementacijskih detalja. Linux i
Windows su platforme sa velikim brojem opcija dok je FreeBSD manje
prilagođen korisniku, ali nudi visoku razinu konfiguracije zbog čega ga
preporučamo korisnicima sa posebnim zahtjevima. OpenVPN tehnologija je vrlo
vjerojatno najbolji izbor za oba spektra korisnika jer dolazi s gotovim
skriptama za postavljanje, ali korisniku na izbor nudi i mnoge dodatne opcije. 
\begin{center}
\begin{tabular}{c | c | c | c }
	 & \textbf{Windws} & \textbf{Linux} & \textbf{FreeBSD} \\
	 \hline

	Windows 10 VPN & * & &  \\
	\hline
        Tinc VPN & * & * & *(nije preporučeno)  \\
	\hline
	SoftEther VPN & * & * & *(nije preporučeno) \\
	\hline
	OpenVPN & * & * & *  \\
	\hline
	LibreSwan VPN & & *  & * \\
\end{tabular}
\end{center}
